% \iffalse
%<*copyright>
%%%%%%%%%%%%%%%%%%%%%%%%%%%%%%%%%%%%%%%%%%%%%%%%%%%%%%%%%%
%% InsDLJS.sty package,             2002-2-2            %%
%% Copyright (C) 2001-2002  D. P. Story                 %%
%%   dpstory@uakron.edu                                 %%
%%                                                      %%
%% This program can redistributed and/or modified under %%
%% the terms of the LaTeX Project Public License        %%
%% Distributed from CTAN archives in directory          %%
%% macros/latex/base/lppl.txt; either version 1 of the  %%
%% License, or (at your option) any later version.      %%
%%%%%%%%%%%%%%%%%%%%%%%%%%%%%%%%%%%%%%%%%%%%%%%%%%%%%%%%%%
%</copyright>
%<package>\NeedsTeXFormat{LaTeX2e}
%<package>\ProvidesPackage{insdljs}
%<package> [2007/07/26 v2.0i Insert Document Level JavaScripts (dps)]
%<*driver>
\documentclass{ltxdoc}
\usepackage[colorlinks,hyperindex]{hyperref}
\pdfstringdefDisableCommands{\let\\\textbackslash}%
\EnableCrossrefs \CodelineIndex
%\OnlyDescription  % comment out for implementation details
\begin{document}
  \GetFileInfo{insdljs.sty}
  \title{The \texttt{insDLJS} Package}
  \author{D. P. Story\\
    Email: \texttt{dpstory@uakron.edu}}
  \date{processed \today}
  \maketitle
  \tableofcontents
  \let\Email\texttt
  \DocInput{insdljs.dtx}
  \PrintIndex
\end{document}
%</driver>
% \fi
%    \section{Introduction}
% This package defines a new environment, \texttt{insDLJS}, used
% for inserting Acrobat JavaScript into a PDF file created from a
% \LaTeX{} source. This package works correctly for users of
% \textsf{pdftex} or \textsf{dvipdfm}. For those that use the
% \textsf{Acrobat Distiller} (specifically, those that use either
% \textsf{dvips} or \textsf{dvipsone} to produce a postscript file,
% which is then distilled), you are required to have Acrobat~5.0 (or
% later).
%
% \section{The \texttt{insDLJS} Environment}
%
% The following is a quick illustration of the use of the new environment.
% \begin{verbatim}
% \documentclass{article}
% \usepackage[pdftex]{hyperref}
% \usepackage[pdftex]{insdljs}
%
% \newcommand\tugHello{Welcome to TUG 2001!}
%
% \begin{insDLJS}[HelloWorld]{mydljs}{My Private DLJS}
% function HelloWorld()
% {
%     app.alert("\tugHello", 3);
% }
% \end{insDLJS}
% \begin{document}
% \begin{Form}     % needed for \PushButton
%
% \section{Test of the \texttt{insDLJS} Package}
%
% % use built in form button of hyperref
% Push \PushButton[name=myButton,
%        onclick={HelloWorld();}]{Button}
%
% \end{Form}
% \end{document}
% \end{verbatim}
% This environment takes three parameters, the first of which is required
% for users of \texttt{dvips} and \texttt{dvipsone} and optional otherwise.
% Within the environment, `|\|', is the escape character, and `|%|' is the comment
% character, as they are in \TeX{} and \LaTeX.
% See the documentation preceding the definition of the
% \texttt{\hyperlink{insDLJS}{insDLJS}} environment.
%
% \section{Debugging}
%
% Another feature of this package is the ability to place debugging markers
% within the JavaScript. When the \texttt{debug} option is used, the markers
% show up in the document level JavaScript; otherwise, they are not written
% to the PDF document. Here is a simple example,
% \begin{verbatim}
%\begin{insDLJS}[HelloWorld]{mydljs}{My Private DLJS}
%function HelloWorld()
%{
%    app.alert("\tugHello", 3);
%    var x = "\\";
%\db console.println("Entered HelloWorld function and x = " + x);\db%
%}
%\end{insDLJS}
%\end{verbatim}
% The last line of the function will appear in the DLJS if compiled under
% the \texttt{debug} option; otherwise, it is removed. Additional discussion
% of this debugging device is given below.
%
% \section{Open Action}
%
% This package also defines an \cs{OpenAction} command to introduce actions that are
% executed when the PDF document is  opened on page~1. The open action command only
% applies to page~1.
% \begin{flushleft}
% \textbf{Usage}
%\begin{verbatim}
%\OpenAction{/S/JavaScript/JS(app.alert("Hello World!");)}
%\end{verbatim}
% \end{flushleft}
% The package defines a |\JS| that takes one argument. The code |\JS{#1}| expands to
% |/S/JavaScript/JS (#1)|, and so the above example may be written as
% \begin{flushleft}
%\begin{verbatim}
%\OpenAction{\JS{app.alert("Hello World!");}}
%\end{verbatim}%
%\end{flushleft}
% Multiple JavaScript commands can be entered. For nice formatting use \cs{r}
% and \cs{t} for carriage return and tab, respectively.
%\goodbreak
%\begin{verbatim}
%\OpenAction{\JS{%
%     app.alert("Hello World!");\r
%     app.alert("Good Day to You!");
%}}
%\end{verbatim}
% The open actions are placed in a token list. Additional actions are added to the token list, so
% you say
%\begin{verbatim}
%\OpenAction{\JS{app.alert("Hello World!");}}
%\OpenAction{\JS{app.alert("Good Day to You!");}}
%\end{verbatim}
% the two messages will appear each time you open page~1 of the document.
%
% In the case of users of the distiller, \textsf{insdljs} uses this mechanism to
% define an open action. To avoid overwriting this definition, the \cs{OpenAction}
% command is necessary in insert additional actions.
%
% See \Nameref{openaction} for more details.
%
% \section{How \textsf{insdljs} Works}
%
% Let me describe in rough terms what goes on behind the scenes. In the discussion
% below, the following example will be used:
%\begin{verbatim}
%\newcommand\tugHello{"Hello World!"}
%\begin{insDLJS}[HelloWorld]{mydljs}{My Private DLJS}
%function HelloWorld()
%{
%    app.alert("\tugHello", 3);
%}
%\end{insDLJS}
%\end{verbatim}
%
% \subsubsection*{For the \texttt{pdftex} and \texttt{dvipdfm} Options}
%
% Both \texttt{pdftex} and \texttt{dvipdfm} have primitives/macros for inserting
% JavaScript at the document level into the PDF document. The work of the package,
% in this case, is to take the JavaScript and place it into the correct form
% for the application (\textsf{pdftex} or \textsf{dvipdfm}).
%
% Material within the \texttt{insDLJS} environment is written verbatim to a file.
% The name of this file is \texttt{mydljs.djs}. The file base name comes from the
% first required parameter of the \texttt{insDLJS} environment, see example above.
% The file extension \texttt{.djs} stands for ``define javascript''.
%
% At begin document, the file \texttt{mydljs.djs} is input back into the document after
% special definitions have already been read. These special definitions are the content
% of the file \texttt{dljscc.def}.  The script is placed in the appropriate construct
% for insertion by the application at the document level.
%
% For more information about \texttt{dljscc.def}, see documentation given in the Section~\ref*{dljscc},
% \Nameref{dljscc}.
%
% \subsubsection*{For the \texttt{dvipsone} and \texttt{dvips} Options}
%
% When you use the \texttt{dvipsone} or \texttt{dvips}, this means you are going to convert
% your document to postscript and distill it to create a PDF document. There is no \texttt{pdfmark}
% construct for inserting document level JavaScripts; however, the recent release of Acrobat~5.0
% gives postscript users an avenue for inserting DLJS.
%
% The material within the \texttt{insDLJS} environment is written verbatim to the file
% \texttt{mydljs.djs}, as above. It is input back into the document where macros are allowed
% to expand, then written back out to another file. The name of this new file is \texttt{mydljs.fdf}.
% The file base name comes from the first required parameter of the \texttt{insDLJS} environment.
% The file extension \texttt{.fdf} stands for ``forms data format''. This is an extension defined
% and recognized by the Acrobat line of products.
%
% The file \texttt{mydljs.fdf} contains the document level JavaScript in a form that
% the Acrobat application (version~5.0 or later) can import. Part of the work of the package
% is to define an open page action. When the PDF document is opened for the first time in
% Acrobat, the JavaScript is imported.  Usually, you open the document following distillation.
% After the \texttt{.fdf} file(s) have been imported, you need to save the document, usually using
% the ``SaveAs'' file option, this saves the JavaScript code with the file. (There is no need for the
% \texttt{.fdf} file(s) at this point.)
%
% \section{Comments on JavaScript}
% \subsection{What is Document Level JavaScript?}
% The document level is a location in the PDF document where script can be stored.
% When the PDF document is opened, the document level functions are scanned, and any
% ``exposed script'' is executed.
%
% Normally, the type of scripts you would place at the document level are
% general purpose JavaScript functions, functions that are called repeatedly or large special
% purpose functions.  Functions at the document level
% are known throughout the document, so they can be called by links, form buttons, page open
% actions, etc.
%
% Variables declared within a JavaScript function have local scope, they are not known outside
% that function. However, if you can declare variables and initialize them at the document level outside
% of a function, these variables will have document wide scope. Throughout the document, the values of these
% global variables are known. For example
%\begin{verbatim}
%\begin{insDLJS}[HelloWorld]{mydljs}{My Private DLJS}
%var myVar = 17;                // defined outside a function, global scope
%function HelloWorld()
%{
%   var x = 3;                 // defined inside a function, local scope
%   app.alert("\tugHello", 3);
%}
% \end{insDLJS}
%\end{verbatim}
% Both the function \texttt{HelloWorld()} and the variable \texttt{myVar} are known throughout
% the document. The function \texttt{HelloWorld()} can be called by a mouse up button action;
% some form field, executing some JavaScript, may access the value of \texttt{myVar} and/or
% change its value.  The variable \texttt{x} is not known outside of the \texttt{HelloWorld()} function.
%
% \subsection{Access and Debugging}
% For those who do not have \textsf{Acrobat}, the application,
% unless you are writing very simple code, writing and debugging
% JavaScript will be very difficult.  From the Acrobat Reader,
% there is not access to the document level JavaScript. You will be
% pretty much writing blind. You can use the debug feature of
% \textsf{insdljs} and insert some debugging code to try to give you
% some insight into what is going wrong. Even so, debugging will be a problem.
%
% Normally, I develop the JavaScript from within Acrobat. The GUI editor does check for
% syntax errors, giving you a chance to correct some simple errors as you go. After I am satisfied
% with my code, I copy it from the editor and paste it into a \texttt{insDLJS} environemnt. This is
% how the JavaScript code of \textsf{exerquiz} was developed.
%
% In my opinion, if you want to develop rather complicated code, having the full Acrobat product
% is a must. (This implies that the Windows or Mac platform is needed!)
%
% \subsection{JavaScript References}
% The JavaScript used by Acrobat consists of the core JavaScript plus Acrobat's JavaScript extensions.
% Acrobat~5.0 uses core JavaScript~1.5, see\newline
% \strut{\small\url{http://developer.netscape.com/docs/manuals/index.html?content=javascript.html}}\newline\noindent
% and documentation of the Acrobat extensions can be found in the ``Acrobat JavaScript Object Specification''
% by Carl Orthlieb and D. P. Story. See\newline
% \strut{\small\url{http://partners.adobe.com/asn/developer/technotes/acrobatpdf.html}}
%
% \section{The \textsf{execJS} Environment}
% This is an environment useful to PDF developers who want to tap into the power of JavaScript.
% To use this environment, the developer needs Acrobat~5.0 or higher, or Acrobat Approval~5.0
% or higher. \textsf{pdftex} or \textsf{dvipdfm} can be used to produce the PDF document, but one of the
% applications (not the Reader) listed above is needed for this environment to do anything.
%
% The \texttt{execJS} is used primarily for post-distillation processing (post-creation processing, in the case of
% \textsf{pdftex} and \textsf{dvipdfm}). The \texttt{execJS} environment can be used, for example, to automatically import
% named icons into the document, which can, in turn, be used for an animation.
%
% The \textsf{execJS} is an environment in which you can write verbatim JavaScript code. This environment
% is a variation on \textsf{insdljs}, it writes a couple of auxiliary files to disk; in particular, the
% environment creates an \texttt{.fdf} file. When the newly produced PDF is loaded for the first time
% into the viewer (Acrobat or Approval, not Reader), the \texttt{.fdf} file generated by the \textsf{execJS}
% environment is imported, and the JavaScript executed. This JavaScript is \emph{not} saved with
% the document.
%
% The environment takes one required argument, the base name of the auxiliary files to be generated.
%
% Many of the useful JavaScript methods have security restrictions,
% but can be executed through a menu event.  A menu item can be
% created with folder-level JavaScript such as the one given below.
% We can use this menu to execute restricted JavaScript.
%\begin{verbatim}
%_MenuProc = function() {;}
%app.addMenuItem({
%    cName: "MenuProc",
%    cUser: "Menu Procedure",
%    cParent: "Tools",
%    cExec: "_MenuProc()",
%    nPos: 0
%});
%\end{verbatim}
% Within the preamble of our document, we can use the \textsf{execJS} environment
% to write some JavaScript to import some named PDF icons into the document.
%\begin{verbatim}
%\begin{execJS}{execjs}
%function importMyIcons ()
%{
%    for ( var i=0; i < 36; i++)
%        this.importIcon("rotate"+i,"animation.pdf",i);
%}
%_MenuProc = importMyIcons;
%app.execMenuItem("MenuProc");
%_MenuProc = function() {;}
%\end{execJS}
%\end{verbatim}
% Once the icons are embedded in the document, they can be used as appearances for button faces.
% An interesting application to this would be a simple animation/slide show.
%
% For more details on this environment, see the section
% \Nameref{execJS}. Also, see the demo file \texttt{execjstst.tex} and \texttt{execjstst.pdf} for
% a short tutorial on the \texttt{execJS} environment, and an animation example, built entirely from
% {\LaTeX} commands.
%
% \paragraph*{Security Note:} Executing arbitrary
%JavaScript through a menu is potentially a security hole. If a
%malicious hacker knows that you have a menu item with the name of
%\texttt{MenuProc}, and that you use the variable
%\texttt{\_MenuProc} in the way described above, the hacker could
%distribute a PDF that executes (security restricted) JavaScript
%through the menu in this way. Therefore, you should select
%\texttt{private names} for \texttt{MenuProc} and \texttt{\_MenuProc}.
%
% \section{The \textsf{defineJS} Environment}
%
%When you create a form element (button, text field, etc.), you sometimes want to attach
%JavaScript.  The \textsf{defineJS} environment aids you in writing your Field level JavaScript.
%It too is a verbatim environment, however, this environment does not write to this, but
%saves the contents in a token register. The contents of the register are used in defining
%a macro that expands to the verbatim listing.
%
% The \textsf{defineJS} environment takes two parameters, the first optional.
% the required parameter is the cmd name of the command you want to define.
% You can use the optional first parameter to modify the verbatim environment,
% as illustrated.  The \textsf{defineJS} is a complete verbatim environment: no escape,
% and no comment characters are defined. You can use the optional parameter to create
% an escape character.  You can pretty much use any character you wish, \emph{except}
% the usual one `\verb+\+', backslash.
%
% The following environment, defines a command \cs{myCmd} what expands to the verbatim
% contents of the environment.
%\begin{verbatim}
%\begin{defineJS}{\myCmd}
% ...
% <JavaScript code>
% ...
%\end{defineJS}
%\end{verbatim}
%
% There is also a \textsf{localJS} environment that makes any definitions within the
% environment local.
%
%The following examples uses some macros from the \textsf{eforms} package:
%\begin{verbatim}
% % Define a text macro that will be expanded from within the
% % defineJS environment
%\def\HelloWorld{Hello World}
% % Now define some JavaScript to execute with a button to be created.
% % We enclose the defineJS and the \pushButton in the localJS environment
% % That way, the macro definitions, \JSA, \JSAAE, \JSAAX will be local to
% % this group
%\begin{localJS}
% % Make @ the escape so we can demonstrate the optional parameter.
%\begin{defineJS}[\catcode`\@=0\relax]{\JSA}
%var sum = 0;
%for (var i = 0; i < 10; i++)
%{
%    sum += i;
%    console.println("@HelloWorld i = " + i );
%}
%console.println("sum = "+sum);
%\end{defineJS}
%\begin{defineJS}{\JSAAE}
%console.println("Enter the button area");
%\end{defineJS}
%\begin{defineJS}{\JSAAX}
%console.println("Exiting the button area");
%\end{defineJS}
%\pushButton[%
%    \A{\JS{\JSA}}
%    \AA{\AAMouseEnter{\JS{\JSAAE}}
%       \AAMouseExit{\JS{\JSAAX}}}
%]{myButton}{30bp}{15bp}
%\end{localJS}
%\end{verbatim}
%\StopEventually{}
% See the section \Nameref{defineJS} for details of these two environments.
% \section{Package Options and Requirements}
%
% \subsection{Package Options}
%
% The options are \texttt{dvipsone}, \texttt{dvips}, \texttt{pdftex} and
% \texttt{dvipdfm}. The default is \texttt{dvipsone}/\texttt{dvips}.
%    \begin{macrocode}
%<*package>
%    \end{macrocode}
%    \begin{macro}{dvipsone}
%    \begin{macro}{dvips}
%    \begin{macro}{textures}
%    \begin{macro}{pdftex}
%    \begin{macro}{dvipdfm}
% Standard driver options.\par\medskip\noindent
% \textbf{Those using Distiller 5.0}
%    \begin{macrocode}
\DeclareOption{dvipsone}{\def\dljs@drivernum{0}%
    \AtBeginDocument{\dvips@marker}}
\DeclareOption{dvips}{\def\dljs@drivernum{0}%
    \AtBeginDocument{\dvips@marker}}
\DeclareOption{textures}{\def\dljs@drivernum{0}%
    \AtBeginDocument{\dvips@marker}}
%    \end{macrocode}
%\textbf{Those not using Distiller}
%    \begin{macrocode}
\DeclareOption{pdftex}{\def\dljs@drivernum{1}}
\DeclareOption{dvipdfm}{\def\dljs@drivernum{2}}
%    \end{macrocode}
%    \end{macro}
%    \end{macro}
%    \end{macro}
%    \end{macro}
%    \end{macro}
% Set the default value of the driver: assume the distiller is used.
%    \begin{macrocode}
%\def\dljs@drivernum{0}
%    \end{macrocode}
% When using pdfmarks, the open action is initiated by the document JavaScript code,
% when there was not document JavaScript, the open action did not get executed.
% This was a bad idea, here is a work around. When the \texttt{insDLJS} environment is used,
% we |\let\dljspresent=y|, and if at the beginning of the document, \cs{dljspresent} is still
% \texttt{n}, we'll issue a special execution of \cs{@OAction}.
%    \begin{macrocode}
\def\dvips@marker{\ifx\dljspresent n\@OAction\fi}
%    \end{macrocode}
% This command sequence will mark the passing of the first open event that is defined. The rest of the open events,
% if any, will be in a \texttt{Next} dictionary.
%    \begin{macrocode}
\let\isOpenAction=n
%    \end{macrocode}
%    \begin{macro}{nodljs}
% Option to cancel the insertion of DLJS, useful for a document meant to be
% printed only, or one that does not use the \texttt{shortquiz} or
% \texttt{quiz} environments, for example.
%    \begin{macrocode}
\DeclareOption{nodljs}{\let\importdljs=n}
\let\importdljs=y
%    \end{macrocode}
%    \end{macro}
% Need to place certain code only once. This switch will make sure of that.
%    \begin{macrocode}
\let\firstdljs=y
\let\dljspresent=n
%    \end{macrocode}
%    \begin{macro}{debug}
% Use this option to help debug DLJS.
%    \begin{macrocode}
\DeclareOption{debug}{\let\dljs@debug=y}
\let\dljs@debug=n
%    \end{macrocode}
%    \end{macro}
%    \begin{macro}{execJS}
% If the execution of JavaScript code is permitted following distillation. The default is no.
%    \begin{macrocode}
\DeclareOption{execJS}{\let\execjs=y}
\let\execjs=n
%    \end{macrocode}
%    \end{macro}
%    \begin{macrocode}
\ProcessOptions
%    \end{macrocode}
% \subsection{Required Packages}
% We need \texttt{hyperref} to provide some fundamental code for the
% various drivers and \texttt{verbatim} to help write verbatim code
% to a file.
%    \begin{macrocode}
\RequirePackage{hyperref}
\RequirePackage{verbatim}
\@ifundefined{dljs@drivernum}{%
    \@ifundefined{eq@drivernum}{%
        \PackageWarning{insdljs}
            {No driver specified, will check for pdftex.}
        \@ifundefined{pdfoutput}{%
        \@ifundefined{@pdfm@mark}{\def\eq@drivernum{2}%
            \def\dljs@drivernum{1}\let\isOpenAction=n
        \typeout{No pdftex, but detected dvipdfm}}
        {\typeout{Neither pdftex nor dvipdfm, assuming dvips/dvipsone.}
        \def\eq@drivernum{0}\def\dljs@drivernum{0}\let\isOpenAction=y}
        }{%
            \ifcase\pdfoutput
                \def\eq@drivernum{1}\def\dljs@drivernum{1}%
                \let\isOpenAction=n\typeout{Pdftex detected.}
            \else
                \def\eq@drivernum{0}\def\dljs@drivernum{0}%
                \let\isOpenAction=y\typeout{Pdftex engine detected,
                but is outputting dvi. Assuming dvips as the driver.}
            \fi
        }%
    }%
    {%
        \begingroup
            \count0 =\eq@drivernum
            \xdef\dljs@drivernum{\ifcase\count0 0\or1\or2\or0\else0\fi}
            \ifnum\dljs@drivernum=0\global\let\isOpenAction=y
                \else\global\let\isOpenAction=n\fi
        \endgroup
    }%
}{}
%    \end{macrocode}
% I've always had problems with the pdftex option and inserting open page actions.
% The problem is that the open page action, which is defined using the token list
% \cs{pdfpageattr} appears not only on the first page, but every page thereafter.
% To solve this problem, I've used \textsf{everyshi}, a nice and useful package by
% Martin Schr\"oder, to set the open page action with \cs{AtNextShipout}. This approach seems to work.
%    \begin{macrocode}
\ifnum\dljs@drivernum=1
    \RequirePackage{everyshi}
\fi
%    \end{macrocode}
%    \section{Main Code}
%
% Before we begin, we need to document the problems that need to be addressed by this package.
%
% \subsection{Complications}
%
% There is a complication with writing JavaScript and \TeX{} together; they both use
% the backslash character, `|\|', as the escape character. The following two tables
% list escaped characters that have special meanings in different situations. These two tables
% were taken from the book \textsl{JavaScript, The Definitive Guide}, by David Flanagan, 4th Edition,
% O'Reilly publishers.
%
%\begin{flushleft}
%\begin{minipage}{\linewidth}
% \textbf{Escape Sequences in String Literals}
%
% \begin{tabular}{ll}
% Sequence & Character represented \\
% \cs{0}   & the NULL character (\cs{u0000}) \\
% \cs{b}   & backspace (\cs{u0008}) \\
% \cs{t}   & horizontal tab (\cs{u0009}) \\
% \cs{n}   & newline (\cs{u000A}) \\
% \cs{v}   & vertical tab (\cs{u000B}) \\
% \cs{f}   & form feed (\cs{u000C}) \\
% \cs{r}   & carriage return (\cs{u000D}) \\
% \cs{"}   & double quote (\cs{u0022}) \\
% \cs{'}   & apostrophe or single quote (\cs{u0027}) \\
% |\\|     & backslash (\cs{u005C})\\
% \cs{xXX} & the Latin-1 character specified by the two hexadecimal digits \texttt{XX} \\
% \cs{uXXXX}& the unicode character specified by the four hexadecimal digits \texttt{XXXX} \\
% \cs{XXX} & the Latin-1 character specified by the octal digits \texttt{XXX}, between \\
%          & $1$ and $377$. \\
% \end{tabular}
%\end{minipage}
%\end{flushleft}
%
%\begin{flushleft}
%\begin{minipage}{\linewidth}
% \textbf{Regular Expression Literal characters}
%
% \begin{tabular}{ll}
% Character & Matches \\
% Alphanumeric & Itself \\
% \cs{0}   & the NULL character (\cs{u0000}) \\
% \cs{t}   & horizontal tab (\cs{u0009}) \\
% \cs{n}   & newline (\cs{u000A}) \\
% \cs{v}   & vertical tab (\cs{u000B}) \\
% \cs{f}   & form feed (\cs{u000C}) \\
% \cs{r}   & carriage return (\cs{u000D}) \\
% \cs{xXX} & the Latin-1 character specified by the two hexadecimal digits \texttt{XX} \\
% \cs{uXXXX}& the unicode character specified by the four hexadecimal digits \texttt{XXXX} \\
% \cs{cXX}   & the control character |^X|
% \end{tabular}
%\end{minipage}
%\end{flushleft}
%
% Again, both JavaScript and \TeX, certain punctuation marks have special meaning; in the case of
% JavaScript, punctuation has a special meaning within regular expressions:
%\begin{flushleft}
%\textbf{Special Punctuation in Regular Expressions}
%\begin{verbatim}
%     ^ $ . * + ? = ! : | \ / ( ) [ ] { }
%\end{verbatim}
%\end{flushleft}
%
% Complications continued. The distiller method (the \texttt{dvipsone} or the
% \texttt{dvips} option), and the \textsf{pdftex}/\textsf{dvipdfm} applications
% handle the backslash differently. The following example illustrates the problem.
% {\small
%\begin{verbatim}
%dvipsone/dvips
%LaTeX Source      As appears in PDF
%var x = "\";  --> var x = "\";  (SyntaxError: Unterminated string literal)
%var x = "\\"; --> var x = "\\"; (Correct)
%\end{verbatim}}
% When we type \texttt{"\string\"}, we are beginning a string (with the first
% double quotes, but then we have a literal \texttt{\string\"}, so we have not
% closed the opening string, hence the error message. We have to type
% `\verb+\\+' to get the single backslash.
% {\small
%\begin{verbatim}
%pdftex/dvipdfm
%LaTeX Source        As appears in PDF
%var x = "\";    --> var x = "";   (\" is the literal ")
%var x = "\\";   --> var x = "\";  (SyntaxError: Unterminated string literal)
%var x = "\\\\"; --> var x = "\\"; (Correct)
%\end{verbatim}}
% These applications write raw PDF code to the \texttt{.pdf} file.
% All special characters need to be escaped. You can see, that \textsf{pdftex}
% and \textsf{dvipdfm} require extra backslashs.
%
% The solution to this problem---the problem of how distiller and the two applications
% handle backslashes---is to define control sequences for all the sequences JavaScript uses
% and to adjust their definitions depending on the driver option. As a result, of these
% background definitions, the JavaScript writer does not worry too much about these details,
% for example
%\begin{verbatim}
%\begin{insDLJS}[HelloWorld]{mydljs}{My Private DLJS}
%function HelloWorld()
%{
%    app.alert("\tugHello", 3);
%    var x = "\\";
%    app.alert("x = " + x);
%}
%\end{insDLJS}
%\end{verbatim}
% So much for the complication.
%
%\subsection{Begin Code}
%
% Some control sequences that are useful for formatting JavaScript.
%    \begin{macro}{\JS}
%    \begin{macro}{\Named}
% A convenience command for writing JavaScript.
%    \begin{macrocode}
\providecommand\JS[1]{/S/JavaScript/JS(#1)}
\providecommand\Named[1]{/S/Named/N/#1}
%    \end{macrocode}
%    \end{macro}
%    \end{macro}
%    \begin{macrocode}
\def\previewMiKTeX{\def\jsR{\string\r}\def\jsT{\string\t}}
{\obeylines %
\gdef\js@@R{\noexpand
}}
\ifnum\dljs@drivernum<2
    \def\jsR{\string\r\eqbs\js@@R}
    \def\defineJSjsR{\string\r\eqbs^^J}
\else
    \def\jsR{\eqbs\js@@R}
    \def\defineJSjsR{\eqbs\js@@R}
\fi
{\catcode`\^^I\active %
\gdef^^I{\noexpand^^I}%
\gdef\js@@T{\noexpand^^I}%
}%  end of \catcode`\^^I
\let\jsT=\js@@T
%    \end{macrocode}
% Need to write to files. This stream will be used for that purpose.
%    \begin{macrocode}
\newwrite\js@verbatim@out
\def\iwvo#1{\immediate\write\js@verbatim@out{#1}}
%    \end{macrocode}
% \subsection{A Macro for Debugging JS}
%    \begin{macro}{\db...\db}
% The \cs{db} macro can be used within the \texttt{insDLJS} to insert
% addition JS commads to help debug the code. Usage:
%\begin{verbatim}
% \db console.println("myVal = " + myVal);\db%
%\end{verbatim}
% Any material (on one line) that is between the two \verb+\db+ will either be
% written to the DLJS (if \cmd{\dljs@debug} expands to \texttt y), or be removed from
% the final output (if \cmd{\dljs@debug} does not expand to \texttt y).
%
% Note that the comment character (|%|) following the terminating \verb+\db+.
% This comment keeps a carriage return from being invoked; if \cmd{dljs@debug} is false
% then this will not create an empty line in your JavaScript.
%    \begin{macrocode}
\def\js@R{\ifcase\dljs@drivernum^^J\else\jsR\fi}
%    \end{macrocode}
% The three spaces (\cs{space}) is meant to align the debug statement, since |\db| takes
% up three spaces.
%    \begin{macrocode}
\def\db#1\db{\ifx\dljs@debug y\space\space\space#1\js@R\fi}
%    \end{macrocode}
%    \end{macro}
% Acrobat (and Reader) expect the JavaScript code will be listed in the JavaScript
% array in sorted form.  \texttt{pdftex} and \texttt{dvipdfm} do not sort this array
% and it would be difficult for \TeX{} to do the sorting; therefore a workaround.
% Each set of DLJS will be numbered in the order in which they are defined. This puts
% them naturally in proper sorted order. The following counter, numbers the JS.
%    \begin{macrocode}
\newcounter{dljs@cnt}
\newcounter{dljssegs}\setcounter{dljssegs}{2}
%    \end{macrocode}
% \subsection{Some Verbatim Write Environments}
% Here is a verbatim write environment, based on an example in the
% \texttt{verbatim} package. One modification: The output stream is
% not opened by this environment, this must be done prior. This is necessary
% to write some other stuff prior to the verbatim write.
%
% This verbatim is used internally to write the \texttt{.djs} files, containing
% the JavaScript.
%    \begin{macrocode}
\newenvironment{js@verbatimwrite}% writes to current \js@verbatim@out
 {%
    \@bsphack
    \let\do\@makeother\dospecials\catcode`\^^M\active
    \def\verbatim@processline{%
        \immediate\write\js@verbatim@out{\the\verbatim@line}}%
    \verbatim@start}{\@esphack}
%    \end{macrocode}
% Unlike the example of verbatimwrite in the \texttt{verbatim} package, we do not
% automatically close the stream when we finish out \texttt{js@verbatimwrite} environment.
% We need to write more to this file, so we have an explicit close.
%    \begin{macrocode}
\def\closejs@verbatim@out{\immediate\closeout\js@verbatim@out}
%    \end{macrocode}
% Same as above, except we expand the line with an \cmd{\edef}.
%
% For the case of \texttt{dvipsone/dvips}, which use the pdfmark operator, we have
% a couple of definitions for escape (esc). It turns out, I find it convenient to have
% two versions, a \cs{eqesc}, and a \cs{eqesci}, the latter one is only used once, the former
% one is used numerous times, see the \texttt{\hyperref[dljscc]{dljscc.def}} file.
%    \begin{macrocode}
\begingroup
\catcode`\@=0 @catcode`@\=12
@gdef@eqbs{\}
@gdef@ccpdfmark{@gdef@eqesc{\}@gdef@eqesci{}}
@endgroup
%    \end{macrocode}
% Here's a definition of left and right braces that will be used in writing JavaScript code.
%    \begin{macrocode}
\begingroup
    \catcode`<=1 \catcode`\>=2 \@makeother\{ \@makeother\}
\gdef\definebraces<\def\{<\eqesc{>\def\}<\eqesc}>>
\endgroup
%    \end{macrocode}
% My own special dos and don'ts. Preserve |\| and \texttt{\%}
%    \begin{macrocode}
\def\eqdospecials{\do\ \do\{\do\}\do\$\do\&%
  \do\#\do\^\do\_\do\~}
%    \end{macrocode}
% The \texttt{jsexpverbatimwrite} environment is used to write a quasi-verbatim
% file. We keep original default definitions of |\| and |%| (they are not included in the
% \cs{eqdospecials} above). The code for \texttt{jsexpverbatimwrite} is based on that found
% in the \texttt{verbatim} package. The trouble is, this package assumes that |\| has been changed
% to catcode 12, other.  Consequently, I had a devil of a time trying to stop the verbatim write
% with |\| as an escape rather than other. I finally decided it was necessary to rewrite some of
% the main macros in \texttt{verbatim} so they would properly stop.  Below are the modifications
% designed to stop when \texttt{*end} encountered rather than \cs{end}. These changes seem to
% work o.k.
%    \begin{macrocode}
\begingroup
\catcode`\~=\active \lccode`\~=`\^^M
\lowercase{\endgroup
    \def\eqverbatim@#1~{\verbatim@@#1*end\@nil}%
    \def\eqverbatim@@#1*end{%
    \verbatim@addtoline{#1}%
    \futurelet\next\eqverbatim@@@}%
    \def\eqverbatim@@@#1\@nil{%
    \ifx\next\@nil
        \verbatim@processline
        \verbatim@startline
        \let\next\eqverbatim@
    \else
        \def\@tempa##1*end\@nil{##1}%
        \@temptokena{*end}%
        \def\next{\expandafter\verbatim@test\@tempa#1\@nil~}%
    \fi \next}%
}%
\def\jsexpverbatimwrite
{% writes to current \js@verbatim@out
    \@bsphack
%    \escapechar=-1%
    \ccpdfmark
    %%
%% This is file `dljscc.def',
%% generated with the docstrip utility.
%%
%% The original source files were:
%%
%% insdljs.dtx  (with options: `copyright,cc4js')
%% 
%%%%%%%%%%%%%%%%%%%%%%%%%%%%%%%%%%%%%%%%%%%%%%%%%%%%%%%%%%
%% InsDLJS.sty package,             2002-2-2            %%
%% Copyright (C) 2001-2002  D. P. Story                 %%
%%   dpstory@uakron.edu                                 %%
%%                                                      %%
%% This program can redistributed and/or modified under %%
%% the terms of the LaTeX Project Public License        %%
%% Distributed from CTAN archives in directory          %%
%% macros/latex/base/lppl.txt; either version 1 of the  %%
%% License, or (at your option) any later version.      %%
%%%%%%%%%%%%%%%%%%%%%%%%%%%%%%%%%%%%%%%%%%%%%%%%%%%%%%%%%%
\def\ckivspace#1{\if\noexpand#1\space\else\expandafter#1\fi}
\def\r{\eqesc r\ckivspace}
\def\t{\eqesc t\ckivspace}
\def\n{\eqesc n\ckivspace}
\def\f{\eqesc f\ckivspace}
\def\v{\eqesc v\ckivspace}
\def\b{\eqesc b\ckivspace}
\def\B{\eqesc B\ckivspace}
\def\d{\eqesc d\ckivspace}
\def\D{\eqesc D\ckivspace}
\def\s{\eqesc s\ckivspace}
\def\S{\eqesc S\ckivspace}
\def\w{\eqesc w\ckivspace}
\def\W{\eqesc W\ckivspace}
\def\x{\eqesc x\ckivspace}
\def\u{\eqesc u\ckivspace}
\def\0{\eqesc0}
\def\1{\eqesc1}
\def\2{\eqesc2}
\def\3{\eqesc3}
\definebraces
\def\({\eqesc\eqesci(}
\def\){\eqesc\eqesci)}
\def\.{\eqesc.}
\def\/{\eqesc/}
\def\[{\eqesc[}
\def\]{\eqesc]}
\def\|{\eqesc|}
\def\+{\eqesc+}
\def\*{\eqesc*}
\def\-{\eqesc-}
\def\?{\eqesc?}
\def\${\eqesc$}
\def\^{\eqesc^}
\def\\{\eqesc\eqesc}
\def\'{\eqesc'}
\catcode`\"=12
\def\"{\eqesc"}
\catcode`\%=12
\def\%{%}
\catcode`\%=14
\catcode`\& = 12
\def\&{\eqesc&}
\endinput
%%
%% End of file `dljscc.def'.
%
% Attempt to redefine some macros from the the verbatim package
    \let\verbatim@=\eqverbatim@
    \let\verbatim@@=\eqverbatim@@
    \let\verbatim@@@=\eqverbatim@@@
% end redefine of verbatim package
    \let\do\@makeother\eqdospecials%
    \catcode`\^^M=\active\catcode`\^^I=12%
    \def\verbatim@processline{%
        \edef\expVerb{\the\verbatim@line}%
        \immediate\write\js@verbatim@out{\expVerb}}%
        \verbatim@start%
}
\def\endjsexpverbatimwrite{\immediate\closeout\js@verbatim@out\@esphack}
%    \end{macrocode}
%    \begin{macro}{\insPath}
% Set this macro in the \textbf{preamble} of your document to
% direct (almost) all auxiliary files of insDLJS to the specified
% path. The macro takes one parameter, a path on your local hard
% disk.  Usage:
%\begin{verbatim}
%\insPath{c:/temp/}
%\end{verbatim}
% Be sure to use only forward slashes, and don't forget to finish
% the string with a final forward slash, as illustrated above.
%
% All \texttt{*.fdf} files are written to this folder. The
% \texttt{.djs} files of any \texttt{insDLJS} environments created
% after the \cmd{\insPath} will also be written to the path. Any
% \texttt{.djs} created by packages loaded earlier (such as
% \textsf{exerquiz}) will be written to the current directory.
%
% This macro may be useful for users of Distiller~5.0, and is of marginal
% value to users of \textsf{pdftex/dvipdfm}.
%    \begin{macrocode}
\let\js@Path=\empty
\def\insPath#1{\def\js@Path{#1}}
%    \end{macrocode}
%    \end{macro}
%
% \subsection{Open Page Actions}\label{openaction}
%
% In order to get automatic insertion of DLJS using Distiller~5.0 or greater, it is necessary
% to use a ``Page Open Action'' for page 1 of the document. This section contains some commands
% for defining an open page action. Originally, these macros were defined only for Distiller
% users (\textsf{dvips} and \textsf{dvipsone} users), later, I extended their use to
% \textsf{pdftex} and \textsf{dvipdfm}.
%
% Here are the macros that actually place the open page action into the PDF document. One for
% the \textsf{Distiller}, one for \textsf{pdftex}, and one for \textsf{dvipdfm}.
%    \begin{macrocode}
\let\@CloseAction\@empty
\def\@OAction@pdfmark{\literalps@out{%
    [ {ThisPage} << /AA << /O << \theFirstAction\space
    \opentoks\@rightDelimiters >> \@CloseAction >> >> /PUT pdfmark}}
%    \end{macrocode}
% For \textsf{pdftex} we use \cs{AtNextShipout}.
%    \begin{macrocode}
\def\@OAction@pdftex{%
    \ifx\isOpenAction y%
        \xdef\pdftexOAction{/AA << /O << \theFirstAction\space
        \opentoks\@rightDelimiters >> \@CloseAction  >>}%
        \AtNextShipout{\pdfpageattr = \expandafter{\pdftexOAction}}%
    \fi
}
\def\@OAction@dvipdfm{\ifx\isOpenAction y%
    \@pdfm@mark{put @thispage << /AA << /O << \theFirstAction\space
    \opentoks\@rightDelimiters >> \@CloseAction >> >>}\fi
}
%    \end{macrocode}
% Now we choose the one appropriate to the user's driver option.
%    \begin{macrocode}
\ifcase\dljs@drivernum
    \let\@OAction=\@OAction@pdfmark
    \gdef\theFirstAction{/S /JavaScript /JS (\the\importfdftoks)}
    \let\isOpenAction=y%
\or
    \let\@OAction=\@OAction@pdftex
    \AtBeginDocument{\@OAction@pdftex}
\or
    \let\@OAction=\@OAction@dvipdfm
    \AtBeginDocument{\@OAction@dvipdfm}
\fi
%    \end{macrocode}
%    \begin{macro}{\OpenAction}
% In order for the DLJS to be inserted, \texttt{insdljs} places an open page action
% on the first page.  If the document author wants his/her own open page action, inserting
% it with the pdfmark operator may lead to unpredictable results.  For the uses of the
% Distiller (dvipsone and dvips users of the distiller), if additional open page action
% is needed on the first page, you can insert this action using the \cmd{\nextOAction} command.
%
%\medskip\noindent Usage
%\begin{verbatim}
%\OpenAction{\JS{app.beep(-1); app.alert("Welcome to my Page!",3);}}
%\end{verbatim}
% The macro takes two parameters, the \cmd{\Next} followed by the desired action, in this
% case it is a JavaScript action.
%
% From the code below, you can see that \cmd{\nextOAction} ends calling itself. If the next token
% is a \cmd{\Next}, additional code is entered into the open page action. Thus, for example, the
% following code is effectively the same as the above example, but the individual lines are
% listed separately in the GUI action dialog of Acrobat.
%\begin{verbatim}
%\OpenAction
%{\JS{app.beep(-1);}}
%\Next{\JS{app.alert("Welcome to my Page!",3);}}
%\end{verbatim}
% In the above example, several open page actions are  ``chained'' together.
%
% You are not restricted to JavaScript actions, for example, you can perform a \texttt{Named}
% action:
%\begin{verbatim}
%\OpenAction{/S /Named /N /Open}
%\end{verbatim}
% There is a |\Named| command that expands to |/S /Named /N#1|, so the above
% example can be written
%\begin{verbatim}
%\OpenAction{/Named{Open}}
%\end{verbatim}
%
%\medskip\noindent \cs{opentoks} accumulates the ``action''.
%    \begin{macrocode}
%\newtoks\opentoks \opentoks={}
\def\opentoks{}
%    \end{macrocode}
% The open action is really not meant for anything too fancy, such as searching using regular
% expressions. Of the special characters, we include only \cs{r} and \cs{t} for formatting
% purposes. If you want to use a regular expression in the open page action (not likely),
% put it at the document level as a JavaScript function, and call that function from the
% open page action.
%    \begin{macrocode}
\def\makespecialJS{%
    \let\r\jsR\let\t\jsT}
\def\@rightDelimiters{}
%    \end{macrocode}
% If the next token is \cs{Next} then we gobble it up, and begin \cs{@OpenAction}
% otherwise we begin \cs{@OpenAction}. This allows the ``chaining'' as illustrated above.
%    \begin{macrocode}
\def\OpenAction{\@ifnextchar\Next
    {\expandafter\@OpenAction\@gobble}{\@OpenAction}}
%    \end{macrocode}
% \cs{@OpenAction}: This is the macro that actually does the work. We differentiate between whether the
% current argument \texttt{\#1} is the first of all open actions, or not. If the former,
% we define it as \cs{theFirstAction}, and reset \cs{isOpenAction}, which keeps track of
% whether there has been a \textit{prior} open action defined. If the latter, we place it in
% the \cs{opentoks} token register.
%    \begin{macrocode}
\def\@OpenAction#1{%
    \ifx\isOpenAction n%
        {\makespecialJS\xdef\theFirstAction{#1}}
        \global\let\isOpenAction=y%
    \else
        \edef\dljstmp{\@rightDelimiters}%
        \xdef\@rightDelimiters{\dljstmp >> }%
        {\makespecialJS\xdef\dljstmp{\opentoks /Next << #1 }}%
        \xdef\opentoks{\dljstmp}%
%        \global\opentoks=\expandafter{\dljstmp}%
    \fi
    \@nextOpenAction
}
%    \end{macrocode}
% As its last act, \cs{@OpenAction} calls \cs{@nextOpenAction}, which is nearly identical to
% \cs{OpenAction}, except if there is no \cs{Next} token, the macro terminates.
%    \begin{macrocode}
\def\@nextOpenAction{\@ifnextchar\Next{\expandafter\@OpenAction\@gobble}{}}
%    \end{macrocode}
%    \end{macro}
% Here is a last example that uses the \texttt{insDLJS} environment and the \cs{OpenAction} command. This code
% will cause the following action: Ten seconds after the document is opened, an annoying
% message is placed on the screen. The annoying message appears only \texttt{once}, otherwise,
% it would really be annoying. Reader (Acrobat)~5.0 is required for this code.
%\begin{verbatim}
%\begin{insDLJS}[DoIt]{doit}{DoIt}
%var timeout;
%var didIt = false;
%var myWelcome = "Welcome to your PDF Page!\r\r You can remove this annoying"
%     +" message by sending \u00A3100 to my Swiss bank account!"
%function startToDoIt()
%{
%    if(!didIt) timeout = app.setInterval("DoIt();", 10000);
%}
%function DoIt()
%{
%    app.beep(-1);
%    app.alert(myWelcome,1);
%    app.clearInterval(timeout);
%    didIt = true;
%}
%\end{insDLJS}
%\OpenAction{/S /JavaScript /JS(startToDoIt();)}
%\end{verbatim}
% See the ``Acrobat JavaScript Object Specification'' for the definitions of some
% of these JavaScript methods.
%
% \subsection{The \texttt{insDLJS} Environments}
%
%    \begin{macro}{insDLJS}\hypertarget{insDLJS}{}
%    \begin{macro}{insDLJS*}
% This is the main environment defined by this package. These environments
% first set some global variables, then set the program flow to driver-dependent
% environments. There are two forms of this environment, the \texttt{insDLJS} and the
% \texttt{insDLJS*}.
%
% The \texttt{insDLJS} is the simplest of the two environments. Any material
% within the environment, eventually ends up in the DLJS section of the
% PDF document
%
% The environment takes the \texttt{<base\_name>} and writes the
% file \texttt{<base\_name>.djs}. This file contains a verbatim
% listing of the JS within the environment, plus some changing of
% catcodes. This file is then input back into the document at
% \cmd{\AtBeginDocument} with the necessary code for
% \textsf{pdftex} and \textsf{dvipdfm} properly place the JS.
%
% The case of \textsf{dvipsone} and \textsf{dvips} is a little different. A
% \texttt{<base\_name>.djs} is written and input back, and a second file
% \texttt{<base\_name>.fdf} is written. This second file is later input
% into the PDF document after distillation.
%
% The syntax of usage for this environment, which takes three
% arguments, is given next.
% \begin{verbatim}
% \begin{insDLJS}[<function>]{<base_name>}{<script_name>}
% <JavaScript functions or exposed code>
% ...
% ...
% \end{insDLJS}
% \end{verbatim}
% where,
% \begin{description}
% \item[\texttt{\#1:}] This optional parameter, \texttt{<function>},
% is \emph{required} for the \texttt{dvipsone} and \texttt{dvips}
% options; otherwise it is ignored. Its value must be the name of
% one of the functions defined in the environment. This is used to
% detect whether the DLJS has already be loaded by Acrobat.
% \item[\texttt{\#2}:] This parameter, \texttt{<base\_name>}, is  an
% alphabetic word with no spaces and limited to eight characters. It
% is used to build the names of auxiliary files and to build the
% names of macros used by the environment.
% \item[\texttt{\#3}:] The \texttt{<script\_name>} of your JavaScript.
% This title will appear in the Documentlevel JavaScript dialog of
% Acrobat.
% \end{description}
% Within the insDLJS environment, there are two types of comment characters:
% (1) a \TeX{} comment (|%|) and (2) a JavaScript comment.   The JavaScript
% comments are `\texttt{//}', a line comment, and `\texttt{/*...*/}' for more
% extensive commenting. These comments will survive and be placed into the
% PDF file. In JavaScript the `|%|' is used as well, use |\%| when you want to use
% the percent character in a JavaScript statement, for example
% |app.alert("\%.2f", 3.14159);|, this statement will appear within your JavaScript
% code as  |app.alert("%.2f", 3.14159);|.
%
%    \begin{macrocode}
\newenvironment{insDLJS}[3][]
{%
    \gdef\detectdljs{#1}\gdef\dljsBase{#2}\gdef\dljsName{#3}%
    \global\let\multisegments=n\setcounter{dljssegs}{2}\global\dljsobjtoks={}%
    \expandafter\ifx\csname dljs\dljsBase\endcsname\relax
        \else\@insjserrDuplicate\fi
    \ifcase\dljs@drivernum
        \let\insert@DLJS=\insert@DLJS@pdfmark
        \let\endinsDLJS=\endinsert@DLJS@pdfmark
        \let\newsegment=\newsegment@pdfmark
        \let\endnewsegment=\endnewsegment@pdfmark
    \or
        \let\insert@DLJS=\insert@DLJS@pdftex
        \let\endinsDLJS=\endinsert@DLJS@pdftex
        \let\newsegment=\newsegment@pdftex
        \let\endnewsegment=\endnewsegment@pdftex
    \or
        \let\insert@DLJS=\insert@DLJS@dvipdfm
        \let\endinsDLJS=\endinsert@DLJS@dvipdfm
        \let\newsegment=\newsegment@dvipdfm
        \let\endnewsegment=\endnewsegment@dvipdfm
    \fi
    \insert@DLJS
}{}
%    \end{macrocode}
%
% The \texttt{insDLJS*} environment can be used to better organize,
% edit and debug your JavaScript.  If you have the full Acrobat
% product, you can open the DLJS edit dialog. There you will see a
% listing of all DLJS contained in the document. When you double
% click on one of the \textsl{script names}, you enter the edit
% window, where you can edit all JavaScript contained under that
% name. Each \texttt{insDLJS} environment creates a new listing
% within this   DLJS dialog; and, each environment creates a
% `\texttt{.djs}' file and possibly  a  \texttt{.fdf}' file. Each
% \texttt{insDLJS*} environment also creates a `\texttt{.djs}' file
% and possibly a `\texttt{.fdf}' file too, but within the
% \texttt{insDLJS*} environment you can create JavaScript under
% different \textsl{script names}.
%
% The syntax is
% \begin{verbatim}
% \begin{insDLJS*}[<function>]{<base_name>}
% \begin{newsegment}{<script_name>}
% <JavaScript functions or exposed code>
% \end{newsegment}
% \begin{newsegment}{<script_name>}
% <JavaScript functions or exposed code>
% \end{newsegment}
% ...
% ...
% <JavaScript functions or exposed code>
% \end{newsegment}
% \begin{newsegment}{<name_name>}
% <JavaScript functions or exposed code>
% \end{newsegment}
% \end{insDLJS*}
% \end{verbatim}
% where,
% \begin{description}
% \item[\texttt{\#1:}] This optional parameter, \texttt{<function>},
% is \emph{required} for the \texttt{dvipsone} and \texttt{dvips}
% options; otherwise it is ignored. Its value must be the name of
% one of the functions defined in the environment. This is used to
% detect whether the DLJS has already be loaded by Acrobat.
% \item[\texttt{\#2}:] This parameter, \texttt{<base\_name>}, is  an
% alphabetic word with no spaces and limited to eight characters. It
% is used to build the names of auxiliary files and to build the
% names of macros used by the environment.
% \item[\texttt{\#3}:] The \texttt{<script\_name>} of your JavaScript.
% This title will appear in the Document level JavaScript dialog of
% Acrobat.
% \end{description}
%    \begin{macrocode}
\newenvironment{insDLJS*}[2][]
{%
    \gdef\detectdljs{#1}\gdef\dljsBase{#2}%
    \global\let\multisegments=y\setcounter{dljssegs}{2}\global\dljsobjtoks={}%
    \expandafter\ifx\csname dljs\dljsBase\endcsname\relax
        \else\@insjserrDuplicate\fi
    \ifcase\dljs@drivernum
        \let\insert@DLJS=\insert@DLJS@pdfmark
        \expandafter\let\csname endinsDLJS*\endcsname=\endinsert@DLJS@pdfmark
        \let\newsegment=\newsegment@pdfmark
        \let\endnewsegment=\endnewsegment@pdfmark
    \or
        \let\insert@DLJS=\insert@DLJS@pdftex
        \expandafter\let\csname endinsDLJS*\endcsname=\endinsert@DLJS@pdftex
        \let\newsegment=\newsegment@pdftex
        \let\endnewsegment=\endnewsegment@pdftex
    \or
        \let\insert@DLJS=\insert@DLJS@dvipdfm
        \expandafter\let\csname endinsDLJS*\endcsname=\endinsert@DLJS@dvipdfm
        \let\newsegment=\newsegment@dvipdfm
        \let\endnewsegment=\endnewsegment@dvipdfm
    \fi
    \insert@DLJS
}{}
\def\@insjserrDuplicate{%
    \typeout{^^J! insdljs Package error.}
    \typeout{! insDLJS environment: On line number \the\inputlineno,}
    \typeout{! the base name `\dljsBase' has already been chosen.}
    \typeout{! A DLJS earlier defined has been overwritten!}
    \typeout{! Choose another name for the first required argument}
    \typeout{! of the insDLJS environment.^^J}
}
%    \end{macrocode}
%    \end{macro}
%    \end{macro}
% \subsection{The \texttt{execJS} Environment}\label{execJS}
% This environment works only for those document authors using Acrobat 5.0 or
% Acrobat Approval. The \texttt{execJS} environment writes a verbatim \texttt{.djs} and
% \texttt{.fdf} files using the same scheme as \texttt{insDLJS} environment.
% The \texttt{.fdf} file is imported into the newly created PDF document and the JavaScript contained
% within the environments are executed.
%
% In order for the JavaScript to be imported and executed, the \texttt{execJS} must be used. The
% JavaScript is executed once, when the document is first opened in Acrobat.
%
% This feature is potentially useful for developmental purposes.
%    \begin{macrocode}
\def\fdfAfterheader
{%
    \iwvo{\string\begingroup}
    \iwvo{\string\makeatletter}
    \iwvo{\string\immediate\string\openout\string\js@verbatim@out=\string\js@Path\space\dljsBase.fdf}%
    \iwvo{\string\begin{jsexpverbatimwrite}}
    \iwvo{\string\firstFDFline}
    \iwvo{1 0 obj}
    \iwvo{<< /FDF << /JavaScript << /Doc 2 0 R /After 3 0 R >> >> >> }
    \iwvo{endobj}
    \iwvo{2 0 obj}
    \iwvo{[ (ExecJS \dljsBase) (var _\dljsBase\space = true;) ] }
    \iwvo{endobj}
    \iwvo{3 0 obj}
    \iwvo{<<>>}
    \iwvo{stream}
%    \iwvo{var \string_\dljsBase;}
%    {\lccode`B=`\{\lowercase{\iwvo{if (typeof \string_\dljsBase == "undefined")B}}}
}
%    \end{macrocode}
%    \begin{environment}{execJS}
% The parameter \#1 is the base name for this environment. The base name will be used to create
% a filename to save the .fdf file under.
%    \begin{macrocode}
\newenvironment{execJS}[1]
{%
    \gdef\detectdljs{\string_#1}\gdef\dljsBase{#1}%
    \global\dljsobjtoks={}%
    \expandafter\gdef\csname dljs\dljsBase\endcsname{}%
    \ifx\importdljs y\ifx\execjs y%
        \ifnum\dljs@drivernum=0
            \addImportAnFDF\importAnFDFTemplate
        \else
%            \xdef\insdljstmp{\importAnFDFTemplate}
            \OpenAction{/S/JavaScript/JS (\importAnFDFTemplate)}%
        \fi
    \fi\fi
    \immediate\openout \js@verbatim@out \js@Path\dljsBase.djs
    \fdfAfterheader
    \js@verbatimwrite
}{%
%    {\lccode`B=`\}\lowercase{\iwvo{B}}}%
%    {\lccode`B=`\}\lowercase{\iwvo{B_\dljsBase = true;}}}%
    \fdfendstreamobj
    \endjs@verbatimwrite
    \fdftrailer
    \closejs@verbatim@out
    \expandafter\xdef\csname\dljsBase OBJ\endcsname{\the\dljsobjtoks}
    \edef\@dljstmp{\noexpand\AtBeginDocument{\noexpand\input{\js@Path\dljsBase.djs}}}%
    \@dljstmp
    \ifx\firstdljs y%
        \AtBeginDocument{\edef\@dljstmp{\importAnFDF}\@dljstmp}%
    \fi
}
%    \end{macrocode}
%    \end{environment}
% \subsection{The \texttt{defineJS} Environment}\label{defineJS}
% When we create a form field that has a JavaScript action attached, it would be nice
% to be able to write verbatim JavaScript code into the action. Due to the limitations
% of \TeX, this is not possible. The other possibility is to create an environment for
% writing JavaScript. This section introduced a couple of environments, the
% \texttt{defineJS} and \texttt{localJS} environments.
%    \begin{environment}{defineJS}
% The \texttt{defineJS} environment takes two parameters, the first one optional.
%\begin{description}
%\item[\ttfamily\#1 = ] This optional parameter can be used to change catcodes
%   before the verbatim read begins.
%\item[\ttfamily\#2 = ] The command to be defined.
%\end{description}
% The material within the environment is read verbatim and a new command by the name of \#2
% is defined.
%    \begin{macrocode}
\newtoks\JStoks
\newenvironment{defineJS}[2][]
{%
    \expandafter\@ifundefined\expandafter{\expandafter\@gobble\string#2}%
        {}{\PackageWarning{insdljs}{The command \string#2 already defined}}%
    \gdef\defineJSArg{#2}\JStoks={}%
    \def\verbatim@processline
        {%
            \xdef\JS@temp{\the\JStoks\the\verbatim@line\defineJSjsR}%\string\r\eqbs^^J}
            \global\JStoks=\expandafter{\JS@temp}%
        }%
    \let\do\@makeother\dospecials\catcode`\^^M\active
    #1%
    \verbatim@start
}{\gdef\eq@JStemp{\expandafter\edef\defineJSArg{\the\JStoks}}%
    \aftergroup\eq@JStemp}
%    \begin{macrocode}
% Silent version of defineJS, can repeatedly be used to redefine the same
% macro.
%    \end{macrocode}
%    \begin{macrocode}
\newenvironment{@defineJS}[2][]
{%
    \gdef\defineJSArg{#2}\JStoks={}%
    \def\verbatim@processline
        {%
            \xdef\JS@temp{\the\JStoks\the\verbatim@line\defineJSjsR}%\string\r\eqbs^^J%
            \global\JStoks=\expandafter{\JS@temp}%
        }%
    \let\do\@makeother\dospecials\catcode`\^^M\active
    #1%
    \verbatim@start
}{\gdef\eq@JStemp{\expandafter\edef\defineJSArg{\the\JStoks}}%
    \aftergroup\eq@JStemp}
%    \end{macrocode}
%    \end{environment}
% An author might want these definitions to be local in definition, so the
% \texttt{localJS} environment is supplied.
%    \begin{environment}{localJS}
% This environment does nothing more than to enclose the material
% in \cs{begingroup} and \cs{endgroup}.
%    \begin{macrocode}
\newenvironment{localJS}[1][]{}{\ifvmode\else\unskip}
%    \end{macrocode}
%    \end{environment}
%
%    \subsection{For \texttt{dvipsone}/\texttt{dvips}}
% The method of automatic insertion of document level JavaScript is to
% write an FDF file (Forms Data Format), then import this FDF into the
% document using the JavaScript method \texttt{Doc.importAnFDF()}.
%
% We break the FDF file into three parts: the \cmd{\fdfheader} (the stuff prior
% to the JavaScript code); the JavaScript code itself; and the
% \cmd{\fdftrailer}, the stuff that follows the code.
%    \begin{macrocode}
\begingroup
\catcode`\%=12
\gdef\firstFDFline{%FDF-1.2}
\gdef\lastFDFline{%%EOF}
\endgroup
\def\fdfheader
{%
    \iwvo{\string\begingroup}
    \iwvo{\string\makeatletter}
    \iwvo{\string\immediate\string\openout\string\js@verbatim@out=\string\js@Path\space\dljsBase.fdf}%
    \iwvo{\string\begin{jsexpverbatimwrite}}
    \iwvo{\string\firstFDFline}
    \iwvo{1 0 obj}
    \iwvo{<< /FDF << /JavaScript << /Doc 2 0 R >> >> >>}
    \iwvo{endobj}
    \iwvo{2 0 obj}
%    \iwvo{[ \csname \dljsBase OBJ\endcsname]}
    \iwvo{[ \string\csname\string\@gobble\space\dljsBase OBJ\string\endcsname]}
    \iwvo{endobj}
}
\def\fdfbeginstreamobj
{%
    \iwvo{\thedljssegs\space 0 obj}
    \iwvo{<<>>}
    \iwvo{stream}
}
\def\fdfendstreamobj{%
    \iwvo{endstream}
    \iwvo{endobj}
}
    \def\fdftrailer{%
    \iwvo{trailer}
    \iwvo{<< /Root 1 0 R >>}
    \iwvo{\string\lastFDFline}
%    \end{macrocode}
% Here we write \texttt{*end{jsexpverbatimwrite}} as a signal for our modified verbatim write
% code to stop.
%    \begin{macrocode}
    \iwvo{*end{jsexpverbatimwrite}}
    \iwvo{\string\endgroup}
}
%    \end{macrocode}
% There may be more than one use of the \texttt{insDLJS}
% environment and we need to be able to import each of the resultant
% FDF files. The \cmd{\importfdftoks} accumulates material to be
% used after all chance for the user to insert JS code has been
% exhausted.
%    \begin{macrocode}
\newtoks\importfdftoks \importfdftoks={}
\newtoks\dljsobjtoks \dljsobjtoks={}
%    \end{macrocode}
% This is a template for detecting whether the DLJS has been imported into
% the document.
%    \begin{macrocode}
\def\importAnFDFTemplate{%
if(typeof \detectdljs\space == "undefined")\jsR\jsT
%    \end{macrocode}
% Beginning with Acrobat 8.1, the \texttt{Doc.importAnFDF} has additional security. We must call
% it through a trusted function which is in the file aeb.js, which resides in the user JavaScript folder.
%\changes{v2.0h}{2006/10/03 }
%{
%   Created aeb.js, which must be installed for users of distiller. The importAnFDF method is now called
%   through a trusted function.
%}
%    \begin{macrocode}
%    this.importAnFDF("\noexpand\js@Path\dljsBase.fdf");\jsR
    ( app.viewerVersion > 8 ) ? aebTrustedFunctions( this, aebImportAnFDF, "\noexpand\js@Path\dljsBase.fdf") : this.importAnFDF("\noexpand\js@Path\dljsBase.fdf");\jsR
}
%    \end{macrocode}
% Add in another template into \cmd{\importfdftoks}.
%    \begin{macrocode}
%\def\addImportAnFDF{%
%    \ifx\importdljs y%
%        \edef\importAnFDFtmp{\the\importfdftoks\importAnFDFTemplate}%
%        \global\importfdftoks=\expandafter{\importAnFDFtmp}%
%    \fi
%}
\def\addImportAnFDF#1{%
    \ifx\importdljs y%
        \let\jsR=\relax\let\jsT=\relax
        \edef\importAnFDFtmp{\the\importfdftoks#1}%
        \global\importfdftoks=\expandafter{\importAnFDFtmp}%
    \fi
}
%    \end{macrocode}
% This is used by \texttt{insert@DLJS@pdfmark} and is placed in the code using
% \cmd{\AtBeginDocument}.
%    \begin{macrocode}
\def\importAnFDF{\ifx\importdljs y\@OAction\fi}
%    \end{macrocode}
% The \texttt{insert@DLJS@pdfmark} environment writes the \cmd{\dljsBase.djs} file
% which is in turn input back in, and rewritten as \cmd{\dljsBase.fdf}. This is
% necessary to give the user a chance to modify the JavaScript code in an
% authorized way.
%    \begin{macrocode}
\newenvironment{newsegment@pdfmark}[1]{%
    \addtocounter{dljssegs}{1}%
    \addtocounter{dljs@cnt}{1}%
    \edef\@dljstmp{\the\dljsobjtoks(#1) \thedljssegs\space 0 R\space}
    \global\dljsobjtoks=\expandafter{\@dljstmp}
    \fdfbeginstreamobj
    \js@verbatimwrite
}{%
    \fdfendstreamobj
    \endjs@verbatimwrite
}
\def\insert@DLJS@pdfmark{%
    \global\let\dljspresent=y%
    \expandafter\gdef\csname dljs\dljsBase\endcsname{}%
    \ifx\importdljs y%
        \addImportAnFDF\importAnFDFTemplate
    \fi
    \immediate\openout \js@verbatim@out \js@Path\dljsBase.djs
    \fdfheader
    \ifx\multisegments n\expandafter\newsegment\expandafter{\expandafter\dljsName\expandafter}\fi
}
\def\endinsert@DLJS@pdfmark{%
    \ifx\importdljs y%
        \ifx\multisegments n\expandafter\endnewsegment\fi
        \fdftrailer
        \closejs@verbatim@out
        \expandafter\xdef\csname\dljsBase OBJ\endcsname{\the\dljsobjtoks}
        \edef\@dljstmp{\noexpand\AtBeginDocument{\noexpand\input{\js@Path\dljsBase.djs}}}%
        \@dljstmp
        \ifx\firstdljs y%
%            \AfterBeginDocument{\edef\@dljstmp{\importAnFDF}\@dljstmp}\global\let\firstdljs=n%
            \AtBeginDocument{\edef\@dljstmp{\importAnFDF}\@dljstmp}\global\let\firstdljs=n%
        \fi
    \fi
}
%    \end{macrocode}
%    \subsection{For \texttt{pdftex}/\texttt{dvipdfm}}
% Again, we break the problem of creating DLJS into four parts: \cmd{\begindljs},
% \cmd{\enddljs}, the code itself, driver dependent material, \cmd{\write@pdftex@obj}
% and \cmd{\write@dvipdfm@obj}.
%
% The following is used by both \texttt{pdftex} and \texttt{dvipdfm}.
%    \begin{macrocode}
\begingroup
\catcode`\@=0 @catcode`@\=12
@gdef@ccpdftex{@gdef@eqesc{\\}@gdef@eqesci{\}}
@endgroup
\def\begindljs
{%
    \iwvo{\string\begingroup}
    {\uccode`c=`\%\uppercase{\iwvo{\string\obeyspaces\string\obeylines\string\global\string\let\string^\string^M=\string\jsR c}}}
    {\escapechar=-1 \lccode`C=`\%\lowercase{\iwvo{\string\\catcode`\string\\"=12C}}}
}
\def\beginseg
{%
    {\lccode`P=`\{\lccode`C=`\%\lowercase{\iwvo{\string\gdef\string\dljs\dljsBase\roman{dljssegs}PC}}}%
}
%    \end{macrocode}
% With \cs{enddsljs}, we now finish the macro definition with a closing right brace, followed by a
% comment, `\texttt\%, and an end of group.
%    \begin{macrocode}
\def\endseg
{%
    {\uccode`c=`\%\uccode`p=`\}\uppercase{\iwvo{pc}}}%
}
\def\enddljs
{%
    \iwvo{\string\endgroup}%
}
%    \end{macrocode}
%    \begin{macrocode}
\def\write@objs
{%
    \iwvo{\begingroup}
    {\lccode`C=`\%\lowercase{\iwvo{\string\ccpdftex C}}}
    {\lccode`C=`\%\lowercase{\iwvo{\string%%
%% This is file `dljscc.def',
%% generated with the docstrip utility.
%%
%% The original source files were:
%%
%% insdljs.dtx  (with options: `copyright,cc4js')
%% 
%%%%%%%%%%%%%%%%%%%%%%%%%%%%%%%%%%%%%%%%%%%%%%%%%%%%%%%%%%
%% InsDLJS.sty package,             2002-2-2            %%
%% Copyright (C) 2001-2002  D. P. Story                 %%
%%   dpstory@uakron.edu                                 %%
%%                                                      %%
%% This program can redistributed and/or modified under %%
%% the terms of the LaTeX Project Public License        %%
%% Distributed from CTAN archives in directory          %%
%% macros/latex/base/lppl.txt; either version 1 of the  %%
%% License, or (at your option) any later version.      %%
%%%%%%%%%%%%%%%%%%%%%%%%%%%%%%%%%%%%%%%%%%%%%%%%%%%%%%%%%%
\def\ckivspace#1{\if\noexpand#1\space\else\expandafter#1\fi}
\def\r{\eqesc r\ckivspace}
\def\t{\eqesc t\ckivspace}
\def\n{\eqesc n\ckivspace}
\def\f{\eqesc f\ckivspace}
\def\v{\eqesc v\ckivspace}
\def\b{\eqesc b\ckivspace}
\def\B{\eqesc B\ckivspace}
\def\d{\eqesc d\ckivspace}
\def\D{\eqesc D\ckivspace}
\def\s{\eqesc s\ckivspace}
\def\S{\eqesc S\ckivspace}
\def\w{\eqesc w\ckivspace}
\def\W{\eqesc W\ckivspace}
\def\x{\eqesc x\ckivspace}
\def\u{\eqesc u\ckivspace}
\def\0{\eqesc0}
\def\1{\eqesc1}
\def\2{\eqesc2}
\def\3{\eqesc3}
\definebraces
\def\({\eqesc\eqesci(}
\def\){\eqesc\eqesci)}
\def\.{\eqesc.}
\def\/{\eqesc/}
\def\[{\eqesc[}
\def\]{\eqesc]}
\def\|{\eqesc|}
\def\+{\eqesc+}
\def\*{\eqesc*}
\def\-{\eqesc-}
\def\?{\eqesc?}
\def\${\eqesc$}
\def\^{\eqesc^}
\def\\{\eqesc\eqesc}
\def\'{\eqesc'}
\catcode`\"=12
\def\"{\eqesc"}
\catcode`\%=12
\def\%{%}
\catcode`\%=14
\catcode`\& = 12
\def\&{\eqesc&}
\endinput
%%
%% End of file `dljscc.def'.
C\the\dljsobjtoks}}}
    \iwvo{\endgroup}
    \iwvo{\string\endinput}%
}
%    \end{macrocode}
%    \subsubsection{\texttt{pdftex} Specific Code}
%    \begin{macrocode}
\newenvironment{newsegment@pdftex}[1]{%
    \addtocounter{dljssegs}{1}%
    \addtocounter{dljs@cnt}{1}%
    \edef\tmp{^^J\string\immediate\string\pdfobj{ << /S /JavaScript /JS
    (\string\dljs\dljsBase\roman{dljssegs}) >> }}
    \edef\@dljstmp{\the\dljsobjtoks\tmp}
    \global\dljsobjtoks=\expandafter{\@dljstmp}
    \edef\tmp{^^J\string\xdef\string\obj\dljsBase\roman{dljssegs}{\string\the\string\pdflastobj\string\space 0 R}}
    \edef\@dljstmp{\the\dljsobjtoks\tmp}
    \global\dljsobjtoks=\expandafter{\@dljstmp}
    \edef\curr@Cnt{\ifnum\arabic{dljs@cnt}<10 0\fi\arabic{dljs@cnt}}
    \edef\dljspdftextmp
    {\the\importfdftoks (\curr@Cnt\space #1) \noexpand\csname obj\dljsBase\roman{dljssegs}\noexpand\endcsname\space}%
    \global\importfdftoks=\expandafter{\dljspdftextmp}%
    \beginseg
    \js@verbatimwrite
}{%
    \endjs@verbatimwrite
    \endseg
}
%    \end{macrocode}
% The main branch of the \texttt{insDLJS} for \texttt{pdftex}. This
% environment writes to the file \cmd{\dljsBase.djs} all the necessary code, then
% is input back into the file using \cmd{\AtBeginDocument}.
%    \begin{macrocode}
\newenvironment{insert@DLJS@pdftex}{%
    \expandafter\gdef\csname dljs\dljsBase\endcsname{}%
    \immediate\openout \js@verbatim@out \js@Path\dljsBase.djs
    \begindljs
    \ifx\multisegments n\expandafter\newsegment\expandafter{\expandafter\dljsName\expandafter}\fi
}{%
    \ifx\multisegments n\expandafter\endnewsegment\fi
    \enddljs
    \write@objs
    \endjs@verbatimwrite
    \closejs@verbatim@out
    \edef\@dljstmp{\noexpand\AtBeginDocument{\noexpand\input{\js@Path\dljsBase.djs}}}%
    \@dljstmp
    \ifx\importdljs y%
        \ifx\firstdljs y%
            \AtEndDocument{\edef\@dljstmp{\setDLJSRef@pdftex}\@dljstmp}
            \global\let\firstdljs=n%
        \fi
    \fi
}
%    \end{macrocode}
% This code places the \texttt{/JavaScript} key-value in the \texttt{/Names} dictionary
% of the PDF document. This code is inserted into the document in \texttt{insert@DLJS@pdftex}
% using \cmd{\AtEndDocument}.
%    \begin{macrocode}
\def\setDLJSRef@pdftex
{%
  \noexpand\immediate\noexpand\pdfobj {%
    << /Names [\the\importfdftoks] >> }%
  \edef\noexpand\objNames{\noexpand\the\noexpand\pdflastobj\space 0 R}%
  \pdfnames {/JavaScript \noexpand\objNames}%
}
%    \end{macrocode}
%    \subsubsection{\texttt{dvipdfm} Specific Code}
% Begin by writing driver-specific code to \cmd{\dljsBase.djs}.
%    \begin{macrocode}
\newenvironment{newsegment@dvipdfm}[1]
{%
    \addtocounter{dljssegs}{1}%
    \addtocounter{dljs@cnt}{1}%
    \edef\tmp{^^J\string\immediate\string\csname\space @pdfm@mark\string\endcsname
        {obj @obj\dljsBase\roman{dljssegs}\space << /S /JavaScript
        /JS (\string\dljs\dljsBase\roman{dljssegs}) >> }}%
    \edef\@dljstmp{\the\dljsobjtoks\space\tmp}
    \global\dljsobjtoks=\expandafter{\@dljstmp}

    \edef\dljspdftextmp
    {\the\importfdftoks (\arabic{dljs@cnt} #1) @obj\dljsBase\roman{dljssegs}\space}%
    \global\importfdftoks=\expandafter{\dljspdftextmp}%
    \beginseg
    \js@verbatimwrite
}{%
    \endjs@verbatimwrite
    \endseg
}
%    \end{macrocode}
% This code places the \texttt{/JavaScript} key-value in the \texttt{/Names} dictionary
% of the PDF document. This code is inserted into the document in \texttt{insert@DLJS@dvipdfm}
% using \cmd{\AtEndDocument}.
%    \begin{macrocode}
\def\setDLJSRef@dvipdfm
{%
  \immediate\@pdfm@mark{obj @objnames
    << /Names [\the\importfdftoks] >> }%
  \@pdfm@mark{put @names
    << /JavaScript @objnames >> }%
}
%    \end{macrocode}
% The main branch of the \texttt{insDLJS} for \texttt{dvipdfm}. This
% environment writes to the file \cmd{dljsBase.djs} all the necessary code, then
% is input back into the file using \cmd{\AtBeginDocument}.
%    \begin{macrocode}
\newenvironment{insert@DLJS@dvipdfm}
{%
    \expandafter\gdef\csname dljs\dljsBase\endcsname{}%
    \immediate\openout \js@verbatim@out \js@Path\dljsBase.djs
    \begindljs
    \ifx\multisegments n\expandafter\newsegment\expandafter{\expandafter\dljsName\expandafter}\fi
}{%
    \ifx\multisegments n\expandafter\endnewsegment\fi
    \enddljs
    \write@objs
    \endjs@verbatimwrite
    \closejs@verbatim@out
    \edef\@dljstmp{\noexpand\AtBeginDocument{\noexpand\input{\js@Path\dljsBase.djs}}}%
    \@dljstmp
    \ifx\importdljs y%
        \ifx\firstdljs y%
            \AtEndDocument{\setDLJSRef@dvipdfm}
            \global\let\firstdljs=n%
        \fi
    \fi
}
%</package>
%    \end{macrocode}
%    \begin{macrocode}
%<*cc4js>
%    \end{macrocode}
%\section{Command Changes for JavaScript}\label{dljscc}
% Regular expressions are a very important part of JavaScript, but they do present
% some problems for \LaTeX.  Ideally, we would like to type JavaScript code into the
% \texttt{insDLJS} environment using standard JavaScript syntax.
% (Whether I am successful remains to be seen.) As a result, some new definitions
% and changes in old definitions are necessary.  All definitions take place within a
% group, so they are unknown outside the \texttt{insDLJS} environment.
%
%\subsection{Regular Expressions}
% Regular expressions can be constructed in two ways, by using (1) a literal text format; or (2)
% using the \texttt{RegExp} constructor function. The literal text format:
%\begin{flushleft}
%\texttt/\textsl{pattern}\texttt/\textsl{flags}
%\end{flushleft}
% The \texttt{RegExp} constructor method:
%\begin{flushleft}\ttfamily
%new RegExp("\textsl{pattern}" [, "\textsl{flags}"])
%\end{flushleft}
% For additional details on regular expressions, see
%\begin{flushleft}\small
%\url{http://developer.netscape.com/docs/manuals/js/core/jsref15/regexp.html}
%\end{flushleft}
% There is a further complication when dealing with regular expressions. To quote from the above
% document:\par\medskip\noindent
% \textbf{Description }
% When using the constructor function, the normal string escape rules (preceding special characters with `|\|' when included in a string) are necessary.
% For example, the following are equivalent:
% \begin{flushleft}
% \ttfamily\obeylines
% re = new RegExp("\string\\w+")
% re = /\string\w+/
%\end{flushleft}
%\subsection{Begin Definitions}
% \subsubsection{Handling of Formatting Sequences}
% The macro \cmd{\eqesc} expands to `|\|' in the case of the distiller and to `|\\|' in the
% case of \textsf{pdftex} or \textsf{dvipdfm}. This group of special characters formats the
% output and they require a different definition than most of the others.
%
% The \cs{ckivspace} (``check for space'') macro sees whether the next token is a space,
% if yes, it absorbs it, if not, it replaces it. I implemented this in order to get
% final output identical to the JavaScript. For example, the following is legal JavaScript
%\begin{verbatim}
% var str = "First line.\r\rSkip a line.     // (1)
%\end{verbatim}
% This is a problem for \TeX. In order to delimit the macro, \TeX{} users need to write the
% above line as
%\begin{verbatim}
% var str = "First line.\r\r Skip a line.    // (2)
%\end{verbatim}
% which then introduces a spurious space into the JavaScript code. The \cs{ckivspace} solves this
% problem, I hope. With the definition of \cs{r} given below, after expansion of the line (2)
% occurs, we have line (1), the space following the second \cs{r} gets absorbed.
%    \begin{macrocode}
\def\ckivspace#1{\if\noexpand#1\space\else\expandafter#1\fi}
%    \end{macrocode}
%    \begin{macro}{\r}
% Matches a carriage return.
%    \begin{macrocode}
\def\r{\eqesc r\ckivspace}
%    \end{macrocode}
%    \end{macro}
%    \begin{macro}{\t}
% Matches a tab.
%    \begin{macrocode}
\def\t{\eqesc t\ckivspace}
%    \end{macrocode}
%    \end{macro}
%    \begin{macro}{\n}
% Matches a linefeed.
%    \begin{macrocode}
\def\n{\eqesc n\ckivspace}
%    \end{macrocode}
%    \end{macro}
%    \begin{macro}{\f}
% Matches a form-feed.
%    \begin{macrocode}
\def\f{\eqesc f\ckivspace}
%    \end{macrocode}
%    \end{macro}
%    \begin{macro}{\v}
% Matches a form-feed.
%    \begin{macrocode}
\def\v{\eqesc v\ckivspace}
%    \end{macrocode}
%    \end{macro}
% \subsubsection{Matching a Word Boundary or Not}
%    \begin{macro}{\b}
% Matches a word boundary, such as a space. (Not to be confused with |[\b]|, which
% matches a backspace.)
%    \begin{macrocode}
\def\b{\eqesc b\ckivspace}
%    \end{macrocode}
%    \end{macro}
% The next group are all defined in the same way, pretty much. The macro \cmd{\eqesc} expands to `|\|'
% in the case of the distiller options, and to `|\\\|' otherwise.
%    \begin{macro}{\B}
% Matches a non-word boundary.
%    \begin{macrocode}
\def\B{\eqesc B\ckivspace}
%    \end{macrocode}
%    \end{macro}
%    \begin{macro}{\cX}
% Where X is a letter from A - Z. Matches a control character in a string.  Doesn't seem to have
% much use in Acrobat JavaScript, will comment this one out.   Usage: \cs{cM}
%    \begin{macrocode}
% \def\doAlpha{\do A\do B\do C\do D\do E\do F%
%   \do G\do H\do I\do J\do K\do K\do M\do N%
%   \do O\do P\do Q\do R\do S\do T\do U\do V\do W\do X\do Y\do Z}
% \def\mkDefns#1{\expandafter\def\csname c#1\endcsname{\eqesc c#1}}
% \let\do=\mkDefns\doAlpha
%    \end{macrocode}
%    \end{macro}
% \subsubsection{Matching Digits or Not}
%    \begin{macro}{\d}
% Matches a digit character. Equivalent to \texttt{[0-9]}.
%    \begin{macrocode}
\def\d{\eqesc d\ckivspace}
%    \end{macrocode}
%    \end{macro}
%    \begin{macro}{\D}
% Matches any non-digit character. Equivalent to |[^0-9]|.
%    \begin{macrocode}
\def\D{\eqesc D\ckivspace}
%    \end{macrocode}
%    \end{macro}
%\subsubsection{Matching Whitespaces or Not}
%    \begin{macro}{\s}
% Matches a single white space character, including space, tab, form feed, line feed.
% Equivalent to |[\f\n\r\t\u00A0\u2028\u2029]|.
%    \begin{macrocode}
\def\s{\eqesc s\ckivspace}
%    \end{macrocode}
%    \end{macro}
%    \begin{macro}{\S}
% Matches a single character other than white space.\newline
% Equivalent to |[^ \f\n\r\t\u00A0\u2028\u2029]|.
%    \begin{macrocode}
\def\S{\eqesc S\ckivspace}
%    \end{macrocode}
%    \end{macro}
%\subsubsection{Matching Alphanumeric Characters or Not}
%    \begin{macro}{\w}
% Matches any alphanumeric character including the underscore. Equivalent to |[A-Za-z0-9\_]|.
%    \begin{macrocode}
\def\w{\eqesc w\ckivspace}
%    \end{macrocode}
%    \end{macro}
%    \begin{macro}{\W}
% Matches any non-word character. Equivalent to |[^A-Za-z0-9_]|.
%    \begin{macrocode}
\def\W{\eqesc W\ckivspace}
%    \end{macrocode}
%    \end{macro}
% \subsubsection{Handling of Character Codes}
%    \begin{macro}{\xXX}
% Matches the character with the code \texttt{XX} (two hexadecimal digits).
% The character with the Latin-1 encoding specified by the two hexadecimal digits \texttt{XX}
% between 00 and FF. For example, \cs{xA9} is the hexadecimal sequence for the copyright
% symbol.
%\begin{flushleft}
%\textbf{Usage}
%\begin{verbatim}
%\x9A
%\end{verbatim}
%if the most significant digit is a number \texttt{[0-9]},
%\begin{verbatim}
%\x A9
%\end{verbatim}
%if the most significant digit is a letter \texttt{[a-fA-F]}
%\end{flushleft}
%    \begin{macrocode}
\def\x{\eqesc x\ckivspace}
%    \end{macrocode}
%    \end{macro}
%    \begin{macro}{\uXXXX}
% In a regular expression search, \cs{uXXXX} matches the character
% with unicode character code \texttt{XXXX} (four hexadecimal
% digits). For example, \cs{u00A9} is the Unicode sequence for the
% copyright symbol. See Unicode Escape Sequences. http://www.unicode.org
%\begin{flushleft}
%\textbf{Usage}
%\begin{verbatim}
%\u00A9
%\end{verbatim}
%if the most significant digit is a number \texttt{[0-9]},
%\begin{verbatim}
%\u F0A9
%\end{verbatim}
%if the most significant digit is a letter \texttt{[a-fA-F]}
%\end{flushleft}
%    \begin{macrocode}
\def\u{\eqesc u\ckivspace}
%    \end{macrocode}
%    \end{macro}
%    \begin{macro}{\XXX}
% The character with the Latin-1 encoding specified by up to three octal digits \texttt{XXX} between
% 0 and 377. For example, \cs{251} is the octal sequence for the copyright symbol.
%
% Since this goes only up to octal 377, it suffices to make only the following four definitions.
%    \begin{macrocode}
\def\0{\eqesc0}
\def\1{\eqesc1}
\def\2{\eqesc2}
\def\3{\eqesc3}
%    \end{macrocode}
%    \end{macro}
%\subsubsection{Other Special Characters}
%    \begin{macro}{braces}
% Braces are a problem when they are unbalanced.  In a regular expression, |\{| works properly, but |\\{| works
% for the distiller options but not otherwise. For example, if I wanted to replace every instance
% of \texttt{\{} with \texttt{(}, we can do this by
%\begin{verbatim}
%     myString = myString.replace( /\{/, "\(");
%\end{verbatim}
% This works for all options, even though the braces are not balanced. However, when the constructor
% function is used, we must double-escape:
%\begin{verbatim}
%     var re = new RegExp("\\{", "g");
%     myString = myString.replace( re, "\(");
%\end{verbatim}
% This works for the distiller options, but not for \texttt{pdftex/dvipdfm}. The reason |\{| works is that
% it is a control sequence, defined by the \cmd{\definebraces} command below. Where as |\\{| consists
% of the control sequence |\\|, which is defined below, followed by a left brace.
% \TeX{} still requires the braces to be balanced.
%
% If you are searching for balanced braces, there should't be a problem, for example,
% the code
%\begin{verbatim}
%     var re = new RegExp("\\{|\\}", "g");
%     myString = myString.replace( re, "\(");
%\end{verbatim}
% works correctly.
%
% To continue the discussion of the problem with searching for
% unbalanced braces, one workaround is to balance out the left
% brace with a right brace in the form of a JavaScript comment.
% This ``fools'' \TeX!  For example,
%\begin{verbatim}
%     var re = new RegExp("\\{", "g");           // }
%     myString = myString.replace( re, "\(");
%\end{verbatim}
% Finally, we have\par\medskip
% \noindent\textbf{The Recommended Workaround:} Use |\\\{| and |\\\}|, e.g.,
%\begin{verbatim}
%     var re = new RegExp("\\\{", "g");
%     myString = myString.replace( re, "\(");
%\end{verbatim}
% This is more verbose than needed, but it works for all options! These comments are only for regular expression
% strings within the constructor function. Note the replacement text, |\(|, is only escaped.
%
% Here we make the definitions for left and right braces. See the definition of \cs{definebraces}
% given earlier.
%    \begin{macrocode}
\definebraces
%    \end{macrocode}
%    \end{macro}
%    \begin{macro}{parentheses}
% Just as braces can give us problems with \TeX, so too, parentheses can give us problems with
% Acrobat JavaScript interpreter.  There seems to be a requirement that parentheses be balanced
% unless they are properly escaped.  Generally, we have no problems in the case of distiller options,
% once again \textsf{pdftex} and \textsf{dvipdfm} are causing (me) problems.
%
% No problems when the regular expression uses the literal string format
%\begin{verbatim}
%    myString = myString.replace(/\(/g, "\[");
%\end{verbatim}
% The above code works for all options. However, for the constructor function we have some problems
% in non-distiller options. If we type instead
%\begin{verbatim}
%    var re = new RegExp("\\(", "g");
%    myString = myString.replace(re, "\[");
%\end{verbatim}
% The above code works for the distiller folks, but not for the
% others. In the case of \textsf{pdftex}, I cannot access the UI to
% the document level JavaScripts, and the script is undefined. I suspect
% the lack of balanced parentheses is the curprit.  For example,
% consider the code
%\begin{verbatim}
%    var re = new RegExp("\\(|\\)", "g");
%    myString = myString.replace(re, "\[");
%\end{verbatim}
% works. I can access the UI to the DLJS from Acrobat, and the script is defined.  Another
% example, we can fool the JavaScript interpreter as we did the \TeX{} compiler:
%\begin{verbatim}
%    var re = new RegExp("\\(", "g");  // )
%    myString = myString.replace(re, "\[");
%\end{verbatim}
% Here I have placed a balancing right parenthesis as a JavaScript comment. Both sets of code
% work correctly.
%\par\medskip
%\noindent\texttt{Recommended Workaround:} As in the case of parentheses, a more verbose version
% of the script works of all options.
%\begin{verbatim}
%    var re = new RegExp("\\\(", "g");
%    myString = myString.replace(re, "\[");
%\end{verbatim}
% This is only for the case of the constructor function.
%    \begin{macrocode}
\def\({\eqesc\eqesci(}
\def\){\eqesc\eqesci)}
%    \end{macrocode}
%    \end{macro}
% The rest of these definitions are for characters that have a
% special meaning when they appear in a regular expression. See
%\begin{flushleft}\small
%\url{http://developer.netscape.com/docs/manuals/js/core/jsref15/regexp.html}
%\end{flushleft}
% for a description of the meaning for the special characters given below (and above).
% To ``escape'' the special meaning of any character, ``escape'' them.
%    \begin{macrocode}
\def\.{\eqesc.}
\def\/{\eqesc/}
\def\[{\eqesc[}
\def\]{\eqesc]}
%    \end{macrocode}
%    \begin{macrocode}
\def\|{\eqesc|}
\def\+{\eqesc+}
\def\*{\eqesc*}
\def\-{\eqesc-}
\def\?{\eqesc?}
\def\${\eqesc$}
\def\^{\eqesc^}
\def\\{\eqesc\eqesc}
%    \end{macrocode}
% The next group of characters have no special meaning in a regular expression, but are using elsewhere
% in JavaScript. These definitions may come in handy.
%    \begin{macrocode}
\def\'{\eqesc'}
%    \end{macrocode}
% The \texttt{german} package makes doublequotes, |"|, active, so within the \texttt{insDLJS}
% environment, we change its catcode to 12.
%    \begin{macrocode}
\catcode`\"=12
\def\"{\eqesc"}
%    \end{macrocode}
% Here, we define `\texttt{\%}' differently. The percent has no special meaning in regular expressions,
% but is used by the \texttt{util.printf()} method.
%    \begin{macrocode}
\catcode`\%=12
\def\%{%}
\catcode`\%=14
%    \end{macrocode}
% The `and' symbol has not special meaning for regular expressions, but is used as a logical `and'.
% Probably don't need this escape.
%    \begin{macrocode}
\catcode`\& = 12
\def\&{\eqesc&}
%    \end{macrocode}
%    \begin{macrocode}
%</cc4js>
%    \end{macrocode}
\endinput
